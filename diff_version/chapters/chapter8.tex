\chapter[Conclusions]{Conclusions} \label{ch:conclusions}

{\em Issue Tracking Systems} have long been used to manage bug-fixes,
enhancements or new features (\ie issues). For a software project to achieve a
sustained success, it has to keep including new exciting features, fixing bugs
and improving the existent functionality. In addition, failing to address
issues, may cause a software project to lose its credibility before its users.

On the other hand, addressed issues may suffer undesirable delays before being
released (\eg delivery delay). In this thesis, we empirically study the
delays that are involved prior the release of addressed issues to end users. In
the remainder of this chapter, we outline the contributions of this thesis and
disclose promising venues for future work.

\section{Contributions and Findings}

\conclusionbox{\textbf{Thesis Statement:} Even though issues are addressed, they
may still suffer delivery delays that software development teams need to manage.
Historical data recorded in software repositories can be used to understand and
estimate delivery delay.}

The overarching goal of this thesis is to understand why addressed issues may
suffer delivery delays after being addressed. To perform our studies, we
leverage data that is recorded in {\em Version Control Systems} and {\em Issue
Tracking Systems} and we survey team members of our subject projects. We answer
the questions that guided our studies below:

\begin{itemize}

	\item \textbf{\textit{Study~1---How frequent is delivery delay?}}
		Delivery delays are frequent in our subject projects (\eg 34\%
		to 98\% of addressed issues were delayed by at least one
		release). In addition, we find that such addressed issues are
		unlikely to be delayed solely because they were addressed close
		to an upcoming release. Finally, we are able to accurately
		identify addressed issues that are likely to suffer from
		prolonged delays (obtaining ROC areas above 0.80)
		(\hyperref[ch:study12]{Chapter}~\ref{ch:study12}). 

	\item \textbf{\textit{Study~2---Do rapid releases reduce delivery
		delays?}} Rapid releases are not silver bullets to reduce the
		delivery delay of addressed issues. Instead, rapid releases may
		improve the consistency of the delivery rate of addressed
		issues. We find that the time at which an issue is addressed
		during a release cycle plays a major role in both types of
		release strategy
		(\hyperref[ch:study34]{Chapter}~\ref{ch:study34}).

	\item \textbf{\textit{Study~3---Why do delivery delays occur?}} Limited
		testing capacity and lack of code review are reasons as to why
		delivery delays occur. Also, the allure of delivering addressed
		issues more quickly is the main motivator to adopt a rapid
		release cycle although such a quicker delivery it is not always
		achieved according to our empirical observations. Integration
		rush and the increased time that is spent on polishing addressed
		issues (during rapid releases) emerge as one of the main
		explanations as to why traditional releases may achieve shorter
		delivery delays (\hyperref[chapter6]{Chapter}\ref{chapter6}).

\end{itemize}

%DIF < Our main findings suggest that delivery delays that are incurred by addressed
%DIF < issues introduce a non-negligible bottleneck in the software development
%DIF < process. Organizations should invest on improving their process (\eg testing or
%DIF < code review processes) to mitigate such delivery delays and provide a more
%DIF < efficient delivery of addressed issues.  
\DIFdelbegin %DIFDELCMD < 

%DIFDELCMD < %%%
\DIFdelend We observe that our \DIFaddbegin \DIFadd{some of our }\DIFaddend quantitative findings are corroborated by our
qualitative findings after performing our studies. According to our statistical
models that are used in \hyperref[ch:study12]{Chapter}~\ref{ch:study12}, the
number of addressed issues that are accumulated in the backlog plays a major
role to estimate the delivery delay of addressed issues. We also observe this
result by analyzing the answers of our participants (see
\DIFdelbegin %DIFDELCMD < \hyperref[th:1]{Theme}%%%
\DIFdel{~\ref{th:1}}\DIFdelend \DIFaddbegin \hyperref[find25]{Finding}\DIFadd{~\ref{find25}}\DIFaddend ). Although controversial between our
participants, it is interesting to note that not all of them state that rapid
releases lead to a quicker delivery of addressed issues (see
\DIFdelbegin %DIFDELCMD < \hyperref[th:8]{Theme}%%%
\DIFdel{~\ref{th:8}}\DIFdelend \DIFaddbegin \hyperref[find32]{Finding}\DIFadd{~\ref{find32}}\DIFaddend ). Indeed, our quantitative
\DIFdelbegin %DIFDELCMD < \hyperref[obs:2]{Observation}%%%
\DIFdel{~\ref{obs:2} }\DIFdelend \DIFaddbegin \hyperref[find18]{Finding}\DIFadd{~\ref{find18} }\DIFaddend corroborates such a perception of our
participants. Finally, the {\em time at which an issue is addressed} (a.k.a.
{\em queue position}) which is one of the most important attributes of our
statistical models in \hyperref[ch:study34]{Chapter}~\ref{ch:study34}, received
the highest importance rates from our participants (see
\hyperref[tbl:factors]{Table}~\ref{tbl:factors}). \DIFaddbegin \DIFadd{In addition, our qualitative
findings help us reach deeper insights about our quantitative findings. For
example, we observe through our models that factors that are related to team
workload may strongly impact delivery delays (see
}\hyperref[find10]{Finding}\DIFadd{~\ref{find10}).  By performing our qualitative
analysis, we gather more details as to why workload may have such an impact. For
instance, team workload may produce a lack of personnel to review code, which
can lead to a delivery delay (see }\hyperref[find34]{Finding}\DIFadd{~\ref{find34}).
Also, the collaboration between teams is important because when an addressed
issue is prioritized differently by different teams, such an issue may suffer
from a delivery delay (see }\hyperref[find28]{Finding}\DIFadd{~\ref{find28}).
}\DIFaddend 

Our performed studies suggest that even though issues are addressed, they may
still suffer delivery delays that software development teams need to manage.
Historical data recorded in software repositories can be used to better
understand and estimate delivery delays. In addition, future research in
Software Engineering should not only consider the required time to develop a new
software functionality when assessing new tools or practices, but also whether
such new tools or practices will likely help the delivery of new software
functionalities.  

\section{Future Work}

The studies that are performed in this thesis pave a way for several future
work possibilities. We outline some venues for future work below. 

\begin{description}

	\item[Replication.] Future work could replicate the studies that are
		performed in this thesis using other large scale software
		projects. Such replication studies are important to reach more
		generalizable conclusions about delivery delays\DIFaddbegin \DIFadd{. For instance,
		one could study the most important factors related to delivery
		delays in the industry rather than open source projects}\DIFaddend .

	\item[Software Quality.] To study the trade-off between delivery delays
		and software quality is important. For example, one could
		empirically investigate whether rapid releases deliver software
		issues with a higher quality despite of the increased delivery
		delay of addressed issues.

	\item[Tooling.] Several tools could be developed to improve the practice
		of Software Engineering. For instance, {\em Issue Tracking
		Systems} could allow the tagging of addressed issues that are
		going to be delayed. The development team could also fill in the
		reasons why an addressed issue is being delayed. Such type of
		features can be used for documentation purposes as well as for
		letting users and contributors more aware as to why an addressed
		issue is being delayed.

	\item[Prediction.] A considerable effort has been invested in the
		prediction of software bugs. Bugs are already known to be
		harmful for software systems, so that their prediction is of
		great importance to avoid unwanted costs. Another possibility of
		future work is the accurate prediction of delivery delays, which
		could allow for a better planing and risk mitigation in software
		projects.

\end{description}

