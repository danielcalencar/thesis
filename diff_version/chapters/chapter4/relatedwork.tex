\section{Related Work} \label{ch4:relatedwork}

In this chapter, we present our studies regarding the general delivery delay of
addressed issues. Hence, we outline related work about possible delays that can
be related to software issues.

Jiang \etal \cite{Jiang2013} studied attributes that could determine the
acceptance and integration of a patch into the Linux kernel. A patch is a record
of changes that is applied to a software system to address an issue. To identify
such attributes, the authors built decision tree models and conducted top node
analysis. Among the studied attributes, developer experience, patch maturity,
and prior subsystem are found to play a major role in patch acceptance and
integration time. Choetkiertikul~\etal~\cite{riskyissues2015a,riskyissues2015b}
study the risk of issues introducing delays that can postpone the shipment of
new releases of a software project. The authors use local attributes (\ie
attributes that can be collected in the issue report itself) and network
attributes (\ie attributes that are extracted from the relationship between
issues) to perform their analyses.

Similar to Jiang \etal\cite{Jiang2013}, we also investigate the integration of
addressed issues.  However, we focus on the delivery delay of issues that are
already addressed rather than the probability to accept a particular patch.
Differently from Choetkiertikul~\etal~\cite{riskyissues2015a,riskyissues2015b},
we study the attributes that may induce addressed issues to be prevented from
delivery rather than the risk of postponing an upcoming release.

