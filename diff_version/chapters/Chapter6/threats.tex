\section{\DIFdelbegin \DIFdel{Threats to Validity}\DIFdelend \DIFaddbegin \DIFadd{Study limitations}\DIFaddend } \label{ch6:threats}

\textbf{\DIFdelbegin \textit{\DIFdel{Internal Validity.}}%DIFAUXCMD
\DIFdelend \DIFaddbegin \textit{\DIFadd{Data limitations.}}\DIFaddend } In our qualitative analysis, we had few
participants (37). However, such an analysis is important to (i) gain insights
from \DIFaddbegin \DIFadd{participants of }\DIFaddend additional projects (ArgoUML and Eclipse) and (ii) better
understand {\em why} delivery delay happens. Moreover, we \DIFdelbegin \DIFdel{choose methods from Grounded Theory
}\DIFdelend \DIFaddbegin \DIFadd{sue Open Coding }\DIFaddend to
perform our qualitative analysis. Although the coding process is performed by
two authors independently and reviewed by a third author, we cannot claim that
we reach all the perspectives that are possible from our questions.

\textbf{\DIFdelbegin \textit{\DIFdel{External Validity.}}%DIFAUXCMD
\DIFdelend \DIFaddbegin \textit{\DIFadd{Generalizability limitations.}}\DIFaddend } External threats are concerned
with our ability to generalize our results. We survey participants from three
different projects (Firefox, ArgoUML, and Eclipse) and perform four follow-up
interviews to gain deeper insights about the responses that we obtain from our
surveys.  Still, we cannot claim that our results are generalizable to other
software projects that are not studied in this work.  Hence, replication of this
work using other projects is required in order to reach more general
conclusions.

