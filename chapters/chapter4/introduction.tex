\section{Introduction}

%\subsection*{Study 1---How Frequent is Delivery Delay?}

Since there is a lack of empirical studies that investigate the frequency of
delivery delays of addressed issues, we perform a study using 20,995 addressed
issues of the ArgoUML, Eclipse, and Firefox projects. Our main goal is to
analyze {\em (i)} how frequent delivery delays occur and {\em (ii)} which
factors may impact delivery delay according to our studied data. Finally, we
also investigate {\em (iii)} what leads to a prolonged delivery delay. In this study, we
address the following RQs:

\begin{itemize}
	\item \textbf{\textit{RQ1: How often are addressed issues prevented from
		being released?}} 34\% to 60\% of addressed issues within
		traditional release cycles (the ArgoUML and Eclipse projects)
		skip at least one release. Furthermore, the delivery of 98\% of
		the addressed issues skip at least one release in the rapidly
		released Firefox project.\\

	\item \textbf{\textit{RQ2: Does the stage of the release cycle 
		impact delivery delay?}} We observe that issues that
		are addressed during more stable stages of a release cycle tend to 
		have a shorter delivery delay. We also observe
		that addressed issues are unlikely to skip releases solely because they
		were addressed near a code freeze period.\\

	\item \textbf{\textit{RQ3: How well can we model the delivery delay of
		addressed issues?}} Our models that are fit to study the
		delivery delay in terms of number of releases obtain AUC values
		of 0.62 to 0.93. Our models that are fit to study the delivery
		delay in terms of number of days obtain $R^2$ values of 0.39 to
		0.65.\\

	\item \textbf{\textit{RQ4: What are the most influential attributes for
		modeling delivery delay?}} We find that the total fixing time
		that is spent per resolver in the release cycle plays an
		influential role in modeling the delivery delay in terms of
		releases of an addressed issue. On the other hand, we find that the
		time at which an issue is addressed and the resolver of the issue
		have a large influence on the delivery delay in terms of days.
		Moreover, attributes that are related to the state of the
		project are the most influential in both types of
		delivery~delay.
%\end{itemize}
%
%
%\subsection{Study~2---What Leads to Prolonged Delivery Delays?}
%
%\begin{itemize}

	\item \textbf{\textit{RQ5: How well can we identify the addressed issues
		that will suffer from a prolonged delivery delay?}} Our models
		outperform na\"{i}ve models like random guessing, achieving AUC
		values of 0.82 to 0.96.\\

	\item \textbf{\textit{RQ6: What are the most influential attributes for
		identifying the issues that will suffer from a prolonged delivery
delay?}} Attributes that are related to the state of the project, such as the
integration workload, the period during which issues are addressed, and the fixing
time that is spent per resolver are the most influential attributes for
identifying the issues that will suffer from a prolonged delivery delay.\\

\end{itemize}


Our results suggest that the total time that is invested per resolver in fixing
the issues of a release cycle has a large influence later in the process of
deliverying addressed issues. Also, the number of issues that are waiting to be
delivered can influence the delivery delay of other addressed issues. Such
results warn us that in addition to the current focus of studies on triaging and
fixing stages of the issue life cycle, the integration and delivery stages
should also be the target of future research and tooling efforts in order to
reduce the time-to-delivery of addressed issues.

\subsection*{Chapter Organization}

This chapter is organized as follows. In
\hyperref[ch4:caseStudy]{Section}~\ref{ch4:caseStudy}, we present the
methodology that is used in our study. In
\hyperref[ch3:results]{Section}~\ref{ch3:results}, we present our obtained
results. In \hyperref[ch4:resultsdiscussion]{Section}~\ref{ch4:discussion}, we
discuss and relate our observations along the studied types of delivery delay.
We perform an exploratory analysis on the backlog of issues of each studied
project in \hyperref[ch4:discussion]{Section}~\ref{ch4:exploratory}. In
\hyperref[ch4:threats]{Section}~\ref{ch4:threats}, we discuss the threats to the
validity of our conclusions, while we draw conclusions
in~\hyperref[ch4:conclusion]{Section}~\ref{ch4:conclusion}.

