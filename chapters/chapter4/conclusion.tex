\section{Conclusions}\label{ch4:conclusion} Once an issue is
addressed, what users and code contributors care most about is when the software
is going to reflect such an addressed issue, \ie when such an addressed issue is
delivered. However, we observed that the delivery of several addressed issues
was prevented for a considerable amount of time. In this context, it is not clear why certain
addressed issues take longer to be integrated than others. We performed
an empirical study of 20,995 issues from the ArgoUML, Eclipse and Firefox
projects. In our study, we:
\\
	\begin{itemize}
		\item despite being addressed well before an upcoming release, 34\% to
			60\% of the addressed issues are not integrated in more than one
			release in the ArgoUML and Eclipse projects. Furthermore, 98\%
			of the Firefox project issues had their delivery delayed by
			at least one release. \\

		\item train random forest models to model the delivery
			delay of an addressed issue. Our models obtain a
			median AUC values between 0.62 to 0.96. Our models
			outperform baseline random and Zero-R models. \\

		\item compute variable importance 
			to understand which attributes are the most important in
			our random forest models to study delivery delay.
			Heuristics that estimate the effort that teams invest in
			fixing issues are the most influential in
			our models to study delivery delay in terms of number of
			releases. \\
			
		\item find that, surprisingly, \textit{priority} and
			\textit{severity} have little impact on our exploratory
			models for delivery delay.
			Indeed, 36\% to 97\% of priority P1 addressed issues
			were delayed by at least one release. \\ 

		\item find that a shorter delivery delay is associated with
			fixes that are performed during more controlled stages
			of a given release cycle.\\

		\item observe that the time at which issues are addressed and the
			resolvers of the issues have great impact on estimating
			the delivery delay of an addressed issue.  Our explanatory
			models obtain $R^2$ values between 0.39 to 0.65. \\

		\item verify that our models that identify addressed issues that
			have a prolonged delivery delay outperform random
			guessing and Zero-R models, obtaining AUC values of 0.82~to~0.96.\\

		\item find that the time at which an issue is addressed (queue
			position), the integration workload (in terms of the
			backlog of addressed issues), and the
			heuristics that estimate the effort that teams invest in
			fixing issues (fixing time per resolver), are the
			most influential attributes for issues that have 
			a prolonged delivery delay. \\

	\end{itemize}

Our work provides insights as to why some addressed issues are integrated prior
to others. Our results suggest that characteristics of the release cycle are the
ones that have the largest impact on delivery delay. Therefore, our findings
highlight the importance of future research and tooling that can support
integrators of software projects. It is important to improve the integration and
delivery stages of a release cycle, since the availability of an addressed issue
in a release is what users and contributors care most about. 

