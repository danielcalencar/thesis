\subsection*{\textbf{RQ5: What are developers' perceptions of shifting to a
rapid release cycle?}}

In this RQ, we study the perceptions of developers about the impact of shifting
to a rapid release cycle. Our findings about these perceptions are organized
along the following themes: {\em management}, {\em delivery}, and {\em
development}. We describe each theme below.\\    

\noindent{\textit{\textbf{Theme~6---Management.}}\theme{th:6} The shift to a rapid
release cycle has a considerable impact on release cycle management. 

The most recurrent theme in this respect is {\em flexibility}$^{(4)}$ to plan
the scope of the releases that should be shipped.  {\em F01}'s opinion is that
rapid releases {\em ``provide a bit more flexibility, since if an important
issue pushed back a less important change and it misses the release cycle, it's
not a huge deal with rapid releases.''} {\em F01}'s observation is supported by
our observation that rapid Firefox releases tend to deliver addressed issues
more consistently (see \hyperref[obs:2]{Observation}~\ref{obs:2}). 

Another perceived advantage of rapid release cycles are the {\em risk
mitigation}$^{(3)}$ and {\em better prioritization}.$^{(3)}$ With respect to
{\em risk mitigation},$^{(3)}$ {\em F07} argues that in rapid release cycles,
the team is {\em ``able to identify issues sooner. It is easier to identify
issues when you have only deployed 3 new commits than 100.''} As for {\em better
prioritization},$^{(3)}$ {\em F19} explains that rapid release cycles {\em
``probably decreases unnecessary delays of the releases because deadline is
closer and developers have to react faster for the pressuring issues.
Non-critical issues gets also pushed back and don't receive useless attention
nor create delays.''} Still on the {\em better prioritization}$^{(3)}$ matter,
{\em F17} adds that rapid releases {\em ``provide a time box in which [the team]
must forecast the top priority work to complete within that time frame.'' }\\

\noindent\textit{\textbf{Theme~7--- Delivery.}}\theme{th:7} The most recurrent perceived
advantage of rapid release cycles is the {\em ``faster delivery''}$^{(33)}$
of new functionalities. When asked about the motivation to use rapid release
cycles, {\em F05} mentions {\em ``increasing speed of getting new features to
users,''} while {\em F06} mentions a similar statement: {\em ``getting new features to
users sooner.''} Interestingly, not all participants that mentioned the time to
deliver new functionalities report that rapid releases always reduce such time.
For {\em F22}, rapid releases {\em ``reduce the time to deliver issues to end
users in some cases, and lengthen them in others.''} More specifically, {\em F24} 
says that {\em ``Low priority issues (new features) take less
time to be delivered, whereas high priority ones (important bugs) take more
time.''} 

Another recurrent perception about rapid releases is the {\em faster user
feedback}$^{(17)}$ due to the constant delivery of new functionalities. For
instance, {\em E29} provides an example that {\em ``you don't find yourself
fixing a bug that you introduced two years ago which the field only discovered on the
release.''}\\

\noindent\textit{\textbf{Theme~8---Development activities.}}\theme{th:8}
We do not observe a specific theme that is recurrent with respect
to development activities. Instead, we observe a broad range of themes that are
cited by the participants. Among such themes, we observe {\em quality},$^{(3)}$
{\em more functionalities},$^{(2)}$ {\em better motivation},$^{(2)}$ and {\em
better prototyping}.$^{(2)}$ Quality should be a measure of success of using
rapid release cycles. According to {\em E26}, {\em ``quality of delivered code
should remain the same or improve''} after switching to rapid releases. Another
way to measure the success of a rapid release cycle is the {\em number of
functionalities}$^{(2)}$ that are completed. {\em E34} states the following: {\em
``I would see if more issues were completely fixed''} as a measure of success. 

Moreover, rapid releases may also impact team members' motivation. For instance,
{\em F06}'s opinion about why to switch to rapid release cycles is {\em ``the
need to motivate the community via more frequent collaboration.''} Finally,
rapid release cycles may also improve prototyping activities. For instance, {\em
P27} argues that, by adopting rapid releases, a development team can {\em ``fix
bugs quickly [and] prototype features, having results in few months.''}

\conclusionbox{The allure of delivering addressed
issues more quickly to users is the most recurrent motivator of switching to a
rapid release cycle. Moreover, the allure of improving management flexibility and quality
of addressed issues are other advantages that are perceived by our participants
with respect to switching to rapid release cycles.}

