\section{Conclusions} \label{sec:conclusion}

In this chapter, we perform a study about the impact that
rapid release cycles have on delivery delays. In our study, we analyze
a total of 72,114 issue reports of 111 traditional releases and 73 rapid
releases of the Firefox project. We obtain the following results:

\begin{itemize}

	\item Although issues tend to be addressed more quickly in the rapid
		release cycle, addressed issues tend to be delivered by 
		consumer-visible releases more quickly in the traditional
		release cycle. However, a rapid release cycle may improve the
		consistency of the delivery rate of addressed issues (see
		\hyperref[find18]{Finding}~\ref{find18}).
	
	\item We observe that the faster delivery of addressed issues in the
		traditional releases is partly due to minor-traditional
		releases. One suggestion for practitioners is that more effort
		should be invested in accommodating minor releases to issues
		that are urgent without compromising the quality of the other
		releases that are being shipped (see
		\hyperref[find19]{Finding}~\ref{find19}).

	\item The triaging time of issues is not significantly different among
		the traditional and rapid releases (see
		\hyperref[find18]{Finding}~\ref{find18}).

	\item The total time that is spent from the issue report date to its
		delivery is not significantly different
		between traditional and rapid releases (see
		\hyperref[find17]{Finding}~\ref{find17}).

	\item In traditional releases, addressed issues are less likely to be
		delayed if they are addressed recently in the backlog. On the
		other hand, in rapid releases, addressed issues are less likely
		to be delayed if they are addressed recently in the current
		release cycle (see \hyperref[find22]{Finding}~\ref{find22}). 

\end{itemize}

This study is the first to empirically check whether rapid releases ship
addressed issues more quickly than traditional releases. Our findings suggest
that there is no silver bullet to deliver addressed issues more quickly.
Instead, rapid releases may increase the consistency of the delivery rate of
addressed issues to end users.

