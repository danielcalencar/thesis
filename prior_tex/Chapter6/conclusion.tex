\section{Conclusions} \label{ch6:conclusion}

In our qualitative study, we survey 37 participants from the Firefox, ArgoUML,
and Eclipse projects. We make the following observations:

\begin{itemize}

      \item  The perceived reasons for delivery delay of addressed issues
      	are primarily related to activities such as development, decision making, team
      	collaboration, and risk management (see
      	\hyperref[th:1]{Themes}~\ref{th:1},~\ref{th:2},~\ref{th:3},~and~\ref{th:5}). 

      \item  The dependency of issues on other projects and team workload are the main
      	perceived reasons to explain our data about delivery delay in
      	general (see \hyperref[th:1]{Theme}~\ref{th:1}).  
      	
      \item  The allure of delivering addressed issues more quickly to users
      	is the most recurrent motivator for switching to a rapid release
      	cycle (see \hyperref[th:7]{Theme}~\ref{th:7}). In addition, the
      	allure of improving management flexibility and quality of
      	addressed issues are other advantages that are perceived by our
      	participants (see \hyperref[th:6]{Themes}~\ref{th:6}~and~\ref{th:8}).

      \item Integration rush and the increased time that is spent on polishing
      	addressed issues (during rapid releases) emerge as one of the main
      	explanations as to why traditional releases may achieve shorter
      	delivery delays (see
      	\hyperref[th:10]{Theme}~\ref{th:10}).

\end{itemize}

Today the Firefox project is often used in support of proposals for moving to a
rapid release cycle throughout many development organizations worldwide. Yet
there has never been any studies that extensively explored the benefits and
challenges (regarding delivery delays) of rapid release cycles on any project
(till our work).  In this regard, our quantitative and qualitative observations
may serve any organization that is interested in adopting a rapid release cycle.
For instance, even though the allure of delivering addressed issues more quickly
is the most recurrent motivator to adopt rapid releases
(\hyperref[th:7]{Theme}~\ref{th:7}), we observe that this often is not achieved
(\hyperref[obs:2]{Observation}~\ref{obs:2}). In summary, our study provides real
observations and offers a wider context of the dis/advantages of adopting a
rapid release strategy.

