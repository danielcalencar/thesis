\section{Introduction} \label{ch6:introduction}

In our prior studies (\hyperref[ch:study12]{Chapter}~\ref{ch:study12}
and~\ref{ch:study34}), we quantitatively investigate the delivery delay of
addressed issues. We perform several statistical analyses based on the data that
is publicly available on the ITSs and VCSs of our subject projects. However, to
reach deeper knowledge as to why delivery delays occur, we survey 37
participants from the ArgoUML, Firefox, and Eclipse projects about the delivery
delay of addressed issues. We also perform follow up interviews with four
participants to get deeper insights about the responses that we receive.  In
\hyperref[st:study4]{Study}~\ref{st:study4}, we qualitatively investigate the
delivery delay of addressed issues to {\em (i)} reach additional insights that
could not be possible by only performing quantitative analysis and {\em (ii)}
verify to which extent our participants agree with our findings from the
quantitative studies. More specifically, we address the following research
questions:

\begin{itemize}

	\item \textbf{\textit{RQ1: What are developers' perceptions as to why
		delivery delays occur?}} The perceived reasons for
		the delivery delay of addressed issues are related to 
		decision making, team collaboration, and risk
		management activities. Moreover, delivery delay will likely lead to
		user/developer frustration according to our participants.\\

	\item \textbf{\textit{RQ2: What are developers' perceptions of shifting
		to a rapid release cycle?}} The allure of delivering addressed
		issues more quickly to users is the most recurrent motivator of
		switching to a rapid release cycle. Moreover, the allure of
		improving the flexibility and quality of addressed issues is another
		advantage that are perceived by our participants.\\

	\item \textbf{\textit{RQ3: To what extent do developers agree with our
		quantitative findings about delivery delay?}} The dependency
		of addressed issues on other projects and team workload are
		the main perceived explanations of our findings about
		delivery delay in general. Integration rush and increased
		time spent on polishing addressed issues (during rapid releases)
		emerge as main explanations as to why traditional releases may
		achieve shorter delivery delays. 
\end{itemize}  

\noindent \textbf{Chapter Organization.} The remainder of this chapter is
organized as follows. In
\hyperref[ch6:studysettings]{Section}~\ref{ch6:studysettings}, we describe the
design of our study. In \hyperref[ch6:results]{Section}~\ref{ch6:results}, we
present the obtained results. In
\hyperref[ch6:threats]{Section}~\ref{ch6:threats}, we disclose the threats to
the validity of our study. Finally, we draw conclusions in
\hyperref[ch6:conclusion]{Section}~\ref{ch6:conclusion}. 

