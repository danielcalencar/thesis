\section{Introduction} \label{sec:introduction}

Within the context of constantly evolving requirements (e.g., in \textit{agile}
development), approaches like eXtreme Programming (XP) and Scrum have arisen to
foster faster software
delivery~\cite{beck2000extreme}.\smartfoot{\url{http://www.scrumguides.org/}}
Those methodologies claim to better embrace a constantly evolving requirements
context by shortening release cycles. Indeed, modern release cycles are on the
order of days or weeks rather than months or years~\cite{baskerville2004short}.
Such rapid releasing enables faster user feedback and a smoother roadmap for
user adoption.

The allure of delivering new features faster has led many large software
projects to shift from a more traditional release cycle (\eg 12-18 months to
ship a major release), to shorter release cycles (\eg weeks). For example,
Google Chrome, Mozilla Firefox, and Facebook teams have each adopted shorter release
cycles~\cite{adams2016saner}. In this chapter, we use the terms \textit{rapid releases} to describe
releases that are shipped using release cycles of weeks or days, and \textit{traditional
releases} to describe releases that are shipped using release cycles of months or years. 

Prior research has investigated the impact of adopting rapid
releases \cite{mantyla2014rapid,souza2014rapid,souzabackout,baysal2011tale,khomh2012faster}.
For example, Khomh~\etal \cite{khomh2012faster} found that bugs that are related to crash
reports tend to be fixed more quickly in the rapid Firefox releases than the traditional
ones. M\"antyl\"a~\etal~\cite{mantyla2014rapid} found that the
Firefox project's shift from a traditional to a rapid release cycle has been
accompanied by an increase in the testing workload.
 
To the best of our knowledge, little prior research has empirically studied the
impact that a shift from a traditional to a rapid release cycle has on the speed
of delivering addressed issues. Such an investigation is important to
empirically check if adopting a rapid release cycle really does lead to the
quicker delivery of addressed issues. In
\hyperref[ch:study12]{Chapter}~\ref{ch:study12}, we study the delay that
happens before the delivery of an addressed issue. We found that 98\% of the
addressed issues in the rapid releases of Firefox were prevented from delivery
in at least one release. Such delayed deliveries hint that even though rapid
releases are consistently shipped every 6 weeks, they may not be delivering
addressed issues as quickly as its proponents purport.

Hence, in this chapter, we compare traditional and rapid release cycles with
respect to delivery delay. We perform a quantitative analysis of 72,114 issue
reports from the Firefox project (34,673 for traditional releases and 37,441 for
rapid releases). These issue reports refer to bugs, enhancements, and new
features~\cite{giuliano2008}. We address the following RQs:
%
%In the second study, we set out to qualitatively
%analyze the delivery delay of addressed issues by surveying 37 participants from
%the Firefox, Eclipse, and ArgoUML projects.

\begin{itemize}

	\item \textbf{\textit{RQ1: Are addressed issues delivered more quickly
		in rapid releases?}} Interestingly, we find that although issues
		are addressed more quickly in rapid releases, they tend to
		require a longer time to be delivered to users.\\

	\item \textbf{\textit{RQ2: Why can traditional releases deliver
		addressed issues more quickly?}} We find that minor-traditional
		releases (\ie releases of smaller scope that are shipped after a
		major version of the software) are a key reason as to why
		addressed issues tend to be delivered more quickly in
		traditional releases. In addition, we find that the length of
		the release cycles are roughly the same between traditional and
		rapid releases when considering both minor and major releases,
		with medians of 40 and 42 days, respectively.\\

	\item \textbf{\textit{RQ3: Did the change in release strategy have an
		impact on the characteristics of delayed issues?}} Our models
		suggest that issues are queued up as a project backlog in
		traditional releases, while issues in rapid releases are queued
		up on a per release basis (\ie a backlog per release cycle).
		Issues that are addressed early either in a project or release
		cycle backlog are less likely to be delayed.
\end{itemize}  

%\subsection{Study 4---Why Do Delivery Delays Occur?}
%
%In this study, we survey 37 developers from the Firefox, Eclipse, and ArgoUML projects to
%study the perceived impact of delivery delay and the shift from a traditional
%to a rapid release cycle. More specifically, we address the following research
%questions:
%
%\begin{itemize}
%
%	\item \textbf{\textit{RQ4: What are developers' perceptions as to why
%		delivery delays occur?}} The perceived reasons for
%		the delivery delay of addressed issues are related to 
%		decision making, team collaboration, and risk
%		management activities. Moreover, delivery delay will likely lead to
%		user/developer frustration according to our participants.\\
%
%	\item \textbf{\textit{RQ5: What are developers' perceptions of shifting
%		to a rapid release cycle?}} The allure of delivering addressed
%		issues more quickly to users is the most recurrent motivator of
%		switching to a rapid release cycle. Moreover, the allure of
%		improving the flexibility and quality of addressed issues is another
%		advantage that are perceived by our participants.\\
%
%	\item \textbf{\textit{RQ6: To what extent do developers agree with our
%		quantitative findings about delivery delay?}} The dependency
%		of addressed issues on other projects and team workload are
%		the main perceived explanations of our findings about
%		delivery delay in general. Integration rush and increased
%		time spent on polishing addressed issues (during rapid releases)
%		emerge as main explanations as to why traditional releases may
%		achieve shorter delivery delays. 
%\end{itemize}  

\noindent \textbf{Chapter Organization.} The remainder of this chapter is
organized as follows. In
\hyperref[sec:studysettings]{Section}~\ref{sec:studysettings}, we describe the
design of this study. In \hyperref[ch4:results]{Section}~\ref{ch4:results}, we
present the obtained results. In
\hyperref[sec:discussion]{Section}~\ref{sec:discussion}, we analyze potential
confounding factors that are related to our results. In
\hyperref[sec:guidelines]{Section}~\ref{sec:guidelines}, we suggest practical
guidelines based on our study. Finally, we disclose the threats to the validity
of our study in \hyperref[sec:threats]{Section}~\ref{sec:threats} and we draw
conclusions in \hyperref[sec:conclusion]{Section}~\ref{sec:conclusion}. 

