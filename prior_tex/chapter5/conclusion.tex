\section{Conclusions} \label{sec:conclusion}

In this chapter, we perform a study about the impact that
rapid release cycles have on delivery delays. In our study, we analyze
a total of 72,114 issue reports of 111 traditional releases and 73 rapid
releases of the Firefox project. We obtain the following results:

\begin{itemize}

	\item Although issues tend to be addressed more quickly in the rapid
		release cycle, addressed issues tend to be integrated into
		consumer-visible releases more quickly in the traditional
		release cycle. However, a rapid release cycle may improve the
		consistency of the delivery rate of addressed issues (see
		\hyperref[obs:2]{Observation}~\ref{obs:2}).
	
	\item We observe that the faster delivery of addressed issues in the
		traditional releases is partly due to minor-traditional
		releases. One suggestion for practitioners is that more effort
		should be invested in accommodating minor releases to issues
		that are urgent without compromising the quality of the other
		releases that are being shipped (see
		\hyperref[obs:3]{Observation}~\ref{obs:3}).

	\item The triaging time of issues is not significantly different among
		the traditional and rapid releases (see
		\hyperref[obs:2]{Observation}~\ref{obs:2}).

	\item The total time that is spent from the issue report date to its
		integration into a release is not significantly different
		between traditional and rapid releases (see
		\hyperref[obs:1]{Observation}~\ref{obs:1}).

	\item In traditional releases, addressed issues are less likely to be
		delayed if they are addressed recently in the backlog. On the
		other hand, in rapid releases, addressed issues are less likely
		to be delayed if they are addressed recently in the current
		release cycle (see \hyperref[obs:6]{Observation}~\ref{obs:6}). 
%
%	\item  The perceived reasons for delivery delay of addressed issues
%		are primarily related to activities such as development, decision making, team
%		collaboration, and risk management (see
%		\hyperref[th:1]{Themes}~\ref{th:1},~\ref{th:2},~\ref{th:3},~and~\ref{th:5}). 
%
%	\item  The dependency of issues on other projects and team workload are the main
%		perceived reasons to explain our data about delivery delay in
%		general (see \hyperref[th:1]{Theme}~\ref{th:1}).  
%		
%	\item  The allure of delivering addressed issues more quickly to users
%		is the most recurrent motivator for switching to a rapid release
%		cycle (see \hyperref[th:7]{Theme}~\ref{th:7}). In addition, the
%		allure of improving management flexibility and quality of
%		addressed issues are other advantages that are perceived by our
%		participants (see \hyperref[th:6]{Themes}~\ref{th:6}~and~\ref{th:8}).
%
%	\item Integration rush and the increased time that is spent on polishing
%		addressed issues (during rapid releases) emerge as one of the main
%		explanations as to why traditional releases may achieve shorter
%		delivery delays (see
%		\hyperref[th:10]{Theme}~\ref{th:10}).

\end{itemize}

This study is the first to empirically check whether rapid releases ship
addressed issues more quickly than traditional releases. Our findings suggest
that there is no silver bullet to deliver addressed issues more quickly.
Instead, rapid releases may increase the consistency of the delivery rate of
addressed issues to end users.

%Although the bulk of our analyses comes mainly from the Firefox project given
%the very unique nature of this project (and the availability of the data), we
%believe that the impact of our findings and work goes well beyond the Firefox
%project. Today the Firefox project is often used in support of proposals for
%moving to a rapid release cycle throughout many development organizations
%worldwide. Yet there has never been any studies that extensively explored the
%benefits and challenges of rapid release cycles on any project (till our work).
%In this regard, our quantitative and qualitative observations may serve any
%organization that is interested in adopting a rapid release cycle. For instance,
%even though the allure of delivering addressed issues more quickly is the most
%recurrent motivator to adopt rapid releases (\hyperref[th:7]{Theme}~\ref{th:7}),
%we observe that this often is not achieved
%(\hyperref[obs:2]{Observation}~\ref{obs:2}). In summary, our study provides real
%observations and offers a wider context of the dis/advantages of adopting a
%rapid release strategy.
