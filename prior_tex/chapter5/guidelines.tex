\section{Practical Suggestions} \label{sec:guidelines}

In this section, we outline suggestions for practitioners and researchers based
on the results of our empirical study.

\begin{description}
	\item \textbf{Small transition.} The choice of adopting a rapid release
		cycle is often motivated by the allure of
		accelerating the delivery of addressed issues. Such a choice needs to be
		carefully rethought. We observe in our empirical study that
		although issues are addressed faster, they tend to wait longer
		to be delivered in the Firefox rapid releases (see
		\hyperref[obs:2]{Observation}~\ref{obs:2}). One
		suggestion for software organizations is to begin the transition
		of release cycles in specific teams or specific products if
		possible. The result of such a small transition could be
		compared with the current development process to test the impact
		of a more rapid release cycle on the
		delivery of addressed issues. \\ 
	\item \textbf{Consistency of delivering addressed issues.} Our empirical
		study suggests that rapid releases can improve the consistency
		of the time to deliver addressed issues (see
		\hyperref[obs:2]{Observation}~\ref{obs:2}). A more consistent
		delivery of addressed issues can be an advantage for the
		software organization, since end users would have a better
		understanding as to when issues will be addressed and delivered. \\
	\item \textbf{Minor releases.} We observe that a large contributor to
		the faster delivery of addressed issues in the Firefox
		traditional releases is due to minor releases (see
		\hyperref[obs:3]{Observation}~\ref{obs:3}). One
		suggestion is that more effort should be invested in
		accommodating minor releases to issues that are urgent without
		compromising the quality of the other releases that are being shipped. \\
\end{description}

