\chapter[Related Work]{Related Work} \label{chapter7}

In this chapter, we survey related research with respect to this thesis. Since
the studies that we perform in this thesis are related to {\em reasons for
delivery delay} (Theme~I) and {\em impact of release strategies on delivery
delay} (Theme~II), we present related work with respect to each studied theme.

Jiang \etal \cite{Jiang2013} studied attributes that could determine the
acceptance and integration of a patch into the Linux kernel. A patch is a record
of changes that is applied to a software system to address an issue. To identify
such attributes, the authors built decision tree models and conducted top node
analysis. Among the studied attributes, developer experience, patch maturity,
and prior subsystem are found to play a major role in patch acceptance and
integration time. 

Choetkiertikul~\etal~\cite{riskyissues2015a,riskyissues2015b} study the risk of
issues introducing delays that can postpone the shipment of new releases of a
software project. The authors use local attributes (\ie attributes that can be
collected in the issue report itself) and network attributes (\ie attributes
that are extracted from the relationship between issues) to perform their
analyses.

Similar to Jiang \etal\cite{Jiang2013}, we also investigate the integration of
addressed issues. However, we focus on the frequency and reasons as to why
delivery delays occur for issues that are already addressed rather than the
probability to accept a particular patch. Differently from
Choetkiertikul~\etal~\cite{riskyissues2015a,riskyissues2015b}, we study the
attributes that may induce addressed issues to be prevented from delivery rather
than the risk of postponing an upcoming release.

Shifting from traditional releases to rapid releases has been shown to have an
impact on software quality and quality assurance activities.
M\"antyl\"a~\etal~\cite{mantyla2014rapid} found that rapid releases have more
tests executed per day but with less coverage. The authors also found that the
number of testers decreased in rapid releases, which increased the test
workload.  Souza~\etal~\cite{souza2014rapid} found that the number of reopened
bugs increased by 7\% when Firefox changed to a rapid release cycle.
Souza~\etal~\cite{souzabackout} found that backout of commits increased when
rapid releases were adopted.  However, they note that such results may be due to
changes in the development process rather than the rapid release cycle---the
backout culture was not widely adopted during the traditional Firefox releases.
We also investigate the shift from traditional releases to rapid releases in
this thesis. However, we analyze delivery delay rather than quality and quality
assurance activities.

It is not clear yet if rapid releases lead to a faster rate of bugs fixes.
Baysal~\etal~\cite{baysal2011tale} found that bugs are fixed faster in Firefox
traditional releases when compared to fixes in the Chrome rapid releases. On the
other hand, Khomh~\etal~\cite{khomh2012faster} found that bugs that are
associated with crash reports are fixed faster in rapid Firefox releases when
compared to Firefox traditional releases.  However, fewer bugs are fixed in
rapid releases, proportionally. Our study corroborates that issues are addressed
more quickly in rapid release cycles, but tend to wait longer to be delivered to
the end users.

Rapid releases may cause users to adopt new versions of the software earlier.
Baysal~\etal \cite{baysal2011tale} found that users of the Chrome browser are
more likely to adopt new versions of the system when compared to traditional
Firefox releases. Khomh~\etal \cite{khomh2012faster} also found that the new
versions of Firefox that were developed using rapid releases were adopted more
quickly than the versions under traditional releases. In this thesis, we
investigate the impact that a shift from traditional to rapid releases has on
delivering addressed issues to users rather than user adoption of new releases.
