% resumo em inglês
\begin{resumo}[Abstract]
 \begin{otherlanguage*}{english}

	 The timely delivery of addressed software issues (i.e., bug fixes,
	 enhancements, and new features) is what drives software development.
	 Previous research has investigated what impacts the time to triage and
	 address (or fix) issues. Nevertheless, even though an issue is
	 addressed, \ie a solution is coded and tested, such an issue may still
	 suffer delay before being delivered to end users. Such delays are
	 frustrating, since end users care most about when an addressed issue is
	 available in the software system (i.e, released). In this matter, there
	 is a lack of empirical studies that investigate why addressed issues
	 take longer to be delivered compared to other issues. In this thesis,
	 we perform empirical studies to understand which factors are associated
	 with the delayed delivery of addressed issues. In our studies, we find
	 that 34\% to 98\% of the addressed issues of the ArgoUML, Eclipse and
	 Firefox projects have their integration delayed by at least one
	 release. Our explanatory models achieve ROC areas above 0.74 when
	 explaining delivery delay. We also find that the workload of
	 integrators and the moment at which an issue is addressed are the
	 factors with the strongest association with delivery delay.  We also
	 investigate the impact of rapid release cycles on the delivery delay of
	 addressed issues. Interestingly, we find that rapid release cycles of
	 Firefox are not related to faster delivery of addressed issues. Indeed,
	 although rapid release cycles address issues faster than traditional
	 ones, such addressed issues take longer to be delivered. Moreover, we find
	 that rapid releases deliver addressed issues more consistently than
	 traditional ones. Finally, we survey 37 developers of the ArgoUML,
	 Eclipse, and Firefox projects to understand why delivery delays occur.
	 We find that the allure of delivering addressed issues more quickly to
	 users is the most recurrent motivator of switching to a rapid release
	 cycle. Moreover, the possibility of improving the flexibility and quality of
	 addressed issues is another advantage that are perceived by our
	 participants. Additionally, the perceived reasons for the delivery
	 delay of addressed issues are related to decision making, team
	 collaboration, and risk management activities. Moreover, delivery delay
	 likely leads to user/developer frustration according to our
	 participants. Our thesis is the first work to study such an important
	 topic in modern software development. Our studies highlight the
	 complexity of delivering issues in a timely fashion (for instance,
	 simply switching to a rapid release cycle is not a silver bullet that
	 would guarantee the quicker delivery of addressed issues).

   \vspace{\onelineskip}
 
   \noindent 
   \textbf{Keywords}: Addressed Issues. Delivery Delay. Mining Software
   Sepositories. Software Maintenance.
 \end{otherlanguage*}
\end{resumo}

